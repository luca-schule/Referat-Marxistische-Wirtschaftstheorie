\documentclass[10pt,a4paper, ngerman]{beamer}
%%% https://docs.google.com/document/d/1SgbuYOFK2A63enCco90Hdt2dBV0MToXd4Bq2IoN8glU/edit#
\usepackage[ngerman]{babel}
\usepackage[T1]{fontenc}
\usepackage[utf8]{inputenc}
\usepackage{varioref}
\usepackage{hyperref}
\usepackage{cleveref}
\usepackage{amsmath}
\usepackage{amsfonts}
\usepackage{amssymb}
\usepackage{makeidx}
\usepackage{graphicx}
\usepackage{csquotes}
\usepackage{listings}
\usepackage{color}
\usepackage{xcolor}
\usepackage[most]{tcolorbox}
\usepackage{amssymb}
\usepackage{lmodern}
\usepackage{verbatim}

% Umlaute und ß in Listings
\lstset{basicstyle=\ttfamily}
\lstset{literate=%
  {Ö}{{\"O}}1
  {Ä}{{\"A}}1
  {Ü}{{\"U}}1
  {ß}{{\ss}}1
  {ü}{{\"u}}1
  {ä}{{\"a}}1
  {ö}{{\"o}}1
}

% Farben für Listings
\definecolor{codegreen}{rgb}{0,0.6,0}
\definecolor{codegray}{rgb}{0.5,0.5,0.5}
\definecolor{codepurple}{rgb}{0.58,0,0.82}
\definecolor{backcolour}{rgb}{0.95,0.95,0.92}
 
\lstdefinestyle{mystyle}{
    backgroundcolor=\color{backcolour},   
    commentstyle=\color{codegreen},
    keywordstyle=\color{magenta},
    numberstyle=\tiny\color{codegray},
    stringstyle=\color{codepurple},
    basicstyle=\footnotesize,
    breakatwhitespace=false,         
    breaklines=true,                 
    captionpos=t,                    
    keepspaces=true,                 
    numbers=left,                    
    numbersep=5pt,                  
    showspaces=false,                
    showstringspaces=false,
    showtabs=false,                  
    tabsize=2,
	language=[LaTeX]{TeX}
}
 
\lstset{style=mystyle}

\definecolor{hbbkBlueish}{HTML}{5e7b87}
\definecolor{hbbkReal}{HTML}{1A4354}
\setbeamercolor{logo}{bg=white}  %controls the color of the logo area
\setbeamercolor{author in sidebar}{fg=white} 
\setbeamercolor{palette primary}{bg=hbbkBlueish,fg=white}
\setbeamercolor{palette secondary}{bg=hbbkBlueish,fg=white}
\setbeamercolor{palette tertiary}{bg=hbbkBlueish,fg=white}
\setbeamercolor{palette quaternary}{bg=hbbkBlueish,fg=white}
\setbeamercolor{structure}{fg=hbbkBlueish} % itemize, enumerate, etc
\setbeamercolor{section in toc}{fg=hbbkReal} % TOC sections
\setbeamertemplate{navigation symbols}{}


\usetheme{PaloAlto}
\renewcommand{\footnotesize}{\small}
\newcommand{\pftn}[1]{\let\thefootnote\relax\footnotetext{\tiny #1}}
\newcommand{\ftn}[2]{\footnote[#1]{\tiny #2}}
\newenvironment{hlbox}{\begin{tcolorbox}[enhanced,colback=white,colframe=white,sharpish corners,fuzzy halo=0.5mm with lightgray]}{\end{tcolorbox}}
\logo{\includegraphics[width=0.155\linewidth]{hbbk-logo}}
\author{Luca Kiebel}

\AtBeginSection{\frame{\frametitle{Gliederung}\tableofcontents[currentsection]}}

\setbeamercovered{invisible}
\author{Luca Hartmann \and Luca Kiebel \and Luca Hülsmann}
\title{Marxistische Wirtschaftstheorie}
%\subtitle{subtitle}
\date{\today}
\institute[HBBK]{Hans-Böckler-Berufskolleg}
\setlength{\itemsep}{10pt}
\begin{document}
\begin{frame}
\titlepage
\end{frame}

\section{Wert- und Geldtheorie}
\begin{frame}
  \frametitle{Wert- und Geldtheorie}
\pftn{http://deacademic.com/dic.nsf/dewiki/925820}
  Karl Marx differenziert Zwei Themen im Handelsbereich:
  \pause
  \begin{itemize}
    \item Wert einer Ware
    \pause
    \item Was ist Geld?
  \end{itemize}
\end{frame}
\subsection{Wert einer Ware}
\begin{frame}
  \frametitle{Wert einer Ware}
\pftn{http://deacademic.com/dic.nsf/dewiki/925820}
  Wie bestimmt sich der Wert einer Ware?
  \pause
  \begin{itemize}
    \item Gebrauchswert
    \pause
    \begin{itemize}
      \item Die Nützlichkeit der Ware bestimmt den Gebrauchswert
    \end{itemize}
    \pause
    \item Tauschwert
    \pause
    \begin{itemize}
      \item Der Aufwand zur Produktion bestimmt den Tauschwert
    \end{itemize}
  \end{itemize}
\end{frame}
\subsection{Was ist Geld?}
\begin{frame}
  \frametitle{Was ist Geld?}
\pftn{http://deacademic.com/dic.nsf/dewiki/925820}
  Marx unterteilte Geld in Vier Aspekte
  \pause
  \begin{itemize}
    \item Geld als Selbstständige Wertform \pause
    \item Geld als Kapital \pause
    \item Geld als Maßstab der Preise \pause
    \item Geld als Ausdruck des Wertmaßes
  \end{itemize}
\end{frame}

\section{Krisentheorie}
\begin{frame}
  \begin{itemize}
      \item Profitrate fällt auf niedrigeres Niveau
      \pause
      \item Kapital wird schneller vermehrt als Arbeitskräfte wachsen
      \pause
      \item mehr Kapital pro Arbeiter, proportional wird nicht mehr verdient
      \pause
      \item Rendite zu gering um Investitionen zu wagen
      \pause
      \item Krisenstreik, da niemand mehr investieren will
      \pause
      \item Krise in der Wirtschaft
  \end{itemize}
\end{frame}

\begin{frame}
  Carl Christian von Weizsäck aus Bonn und Lawrence Summers (damaliger Finanzminister USA)
  argumentierteny, dass die Menschheit zu viel investiert hatte,
  so dass es keine rentablen Investitionsprojekte mehr gab.
\end{frame}

\section{Das Kapital}
\begin{frame}[fragile]{Das Kapital}{Was ist Kapital?}
\pftn{https://de.wikipedia.org/wiki/Marxistische\_Wirtschaftstheorie\#Geld\_als\_Kapital}
\begin{itemize}
	\item Drei Arten:
	\begin{enumerate}
		\item Gegenstand
		\item Prozess
		\item Produktionsverhältnis
	\end{enumerate}
\end{itemize}
\end{frame}

\begin{frame}[fragile]{Das Kapital}{Reproduktionsprozesse}
\begin{itemize}
	\item Unternehmer: \(G - W ... P ... W' - G'\)\ftn{1}{https://de.wikipedia.org/wiki/Kapital} \pause
	\item Verbraucher: \(A - G - W\)\ftn{2}{https://de.wikipedia.org/wiki/Marxistische\_Wirtschaftstheorie\#Geld\_als\_Kapital}
\end{itemize}
\end{frame}

\subsection{Die Zusammensetzung des Kapitals}
\begin{frame}{Das Kapital}{Die Zusammensetzung des Kapitals}
\pftn{https://de.wikipedia.org/wiki/Kapital}
\begin{itemize}
	\item Kauf von Arbeitskraft
	\item Kauf von Produktionsmitteln
\end{itemize}
\end{frame}

\subsection{Die Reproduktion des Kapitals}
\begin{frame}{Die Reproduktion des Kapitals}{Einfache Reproduktion}
\pftn{https://de.wikipedia.org/wiki/Marxistische\_Wirtschaftstheorie\#Die\_Reproduktion\_des\_Kapitals}
\begin{itemize}
	\item Privater Gebrauch des Gewinns
	\item Gleichbleibende Entwicklungsstufe
\end{itemize}
\end{frame}

\begin{frame}{Die Reproduktion des Kapitals}{Erweiterte Reproduktion}
\pftn{https://de.wikipedia.org/wiki/Marxistische\_Wirtschaftstheorie\#Die\_Reproduktion\_des\_Kapitals}
\begin{itemize}
	\item Erweiterung des Geschäfts
	\item Erhöhung der technischen Entwicklungsstufe
\end{itemize}
\end{frame}

\begin{frame}{Die Reproduktion des Kapitals}{Formen der Erweiterung der Reproduktion}
\pftn{https://de.wikipedia.org/wiki/Marxistische\_Wirtschaftstheorie\#Die\_Reproduktion\_des\_Kapitals}
\begin{itemize}
	\item Konzentration
	\begin{itemize}
		\item \(C+c=C'\)
	\end{itemize} \pause
	\item Zentralisation
	\begin{itemize}
		\item \(C_{1}+C_{2}+\ldots =C'\)
	\end{itemize}
\end{itemize}
\end{frame}

\section{Kritik der politischen Ökonomie}
\begin{frame}
  \begin{itemize}
    \item von August 1858 bis Januar 1859 verfasst
    \pause
    \item Berechnen von Wert und Waren als Gebrauchswert und Tauschwert
    \pause
    \item Zusammenfassung von konkreter und abstrakter Arbeit
    \pause
    \item Arbeitszeit als quantitatives Maß der Arbeiter
    \pause
    \item o.ä. Wertfestellungen, z.B. von Edelmetallen
  \end{itemze}
\end{frame}

\section{Preisarten}



\end{document}
